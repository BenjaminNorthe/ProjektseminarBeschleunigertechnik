\chapter*{Abk\"urzungsverzeichnis}
\begin{tabular}{p{0.2\textwidth} p{0.7\textwidth}}
\textbf{Abk\"urzung} & \textbf{Beschreibung} \\
 & \\
MA & magnetic alloy \\
RK & Ringkern \\
<<<<<<< HEAD
GSI & GSI Helmholtzzentrum f\"ur Schwerionenforschung GmbH\\
=======
GSI & Gesellschaft f\"ur Schwerionenforschung \\
NA & Network Analyzer\\
>>>>>>> 699c838b2c1928b020b3f7dfead72a908fac4304
KS & Kurzschluss \\
BNC & Bayonet Neill Concelman: F\"ur Oszilloskope verwendeter Koaxialstecker \\
CST & Computer Simulation Technology \\
PE & Polyethylen \\
RLC & Netzwerk aus Widerstand, Induktivit\"at und Kapazit\"at \\
ESB & Ersatzschaltbild\\
\end{tabular}

\cleardoublepage

\chapter*{Symbolverzeichnis}

\begin{tabular}{p{0.2\textwidth} p{0.7\textwidth}}
\textbf{Symbol} & \textbf{Beschreibung und Einheit}\\
 & \\
$\omega$ & Kreisfrequenz \\

$Z_{rk}$ & Impedanz des MA-Ringkerns \\
$L_{rk}$ & Induktivit\"at des MA-Ringkerns \\
$R_{rk}$ & Wirkwiderstand des MA-Ringkerns \\
$\underline{\mu}_r$ & Komplexe, dissipative Permeabilit\"atskonstante\\
% $\mu^{\prime}$ & \\
% $\mu^{\prime\prime}$ & \\
$N$ & Anzahl der Ringkerne\\
$d$ & Ringkerndicke\\
$r_i$ & Innendurchmesser eines Ringkerns \\
$r_o$ & Au\ss{}endurchmesser eines Ringkerns \\
$\underline{\varepsilon}_r$ & Komplexe, dissipative Dielektrizit\"atskonstante\\
$\underline{Z}_{ges}$ & Gemessene Impedanz an der Testbox \\
$R_{box}$ & Wirkwiderstand des RLC-Modells der Testbox \\
$L_{box}$ & Induktivit\"at des RLC-Modells der Testbox \\
$C_{box}$ & Kapazit\"at des RLC-Modells der Testbox \\
$a_{max}$ & Maximale relative Abweichung eines Messparameters \\
$a_{percent}$ & Maximale relative Abweichung eines Messparameters(Prozent) \\


\end{tabular}
