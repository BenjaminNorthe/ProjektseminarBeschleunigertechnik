\section{Einfluss der Anzahl der Kurzschl\"usse}
F\"ur diese Analyse wurden Kurzschl\"usse mittels Torusringen um den Ringkern erzeugt. Dabei wurde sowohl die Anzahl, als auch die Position variiert. Abbildung~\ref{fig:AnzahlKs} zeigt die Einfl\"usse.
\begin{figure}[h]
	\centering
	\begin{tikzpicture}
		\begin{axis}[ymode = log, width=0.85\textwidth, height = 0.5\textwidth, xmin = 0.1, xmax = 100, xlabel=Frequenz in MHz, ylabel=Realteil der Impedanz der Kavit\"at in Ohm, xticklabel style={/pgf/number format/fixed,/pgf/number format/precision=5}, every axis plot/.append style={thick},every axis legend/.append style={at={(0.02,0.97)},anchor=north west}, grid=both, cycle list name=color list]
			\addplot table[x index=0,y index=1,mark=none] {../Inputfiles/plotData/Box.txt};
			\addplot table[x index=0,y index=1,mark=none] {../Inputfiles/plotData/Rk.txt};
			\addplot table[x index=0,y index=1,mark=none] {../Inputfiles/plotData/Rk1Ks0.txt};
			\addplot table[x index=0,y index=1,mark=none] {../Inputfiles/plotData/Rk4Ks90.txt};
			\addplot table[x index=0,y index=1,mark=none] {../Inputfiles/plotData/Rk24Ks15.txt};
			\legend{leere Box, Box mit Ringkern, 1 Kurzschluss, 4 Kurzschl\"usse (90 Grad versetzt),  24 Kurzschl\"usse}
		\end{axis}
	\end{tikzpicture}
	\caption{Verhaltend der Box ohne Ringkern im Vergleich zur Box mit Ringkern, sowie mit mehreren Kurzschl\"ussen}
	\label{fig:AnzahlKs}
\end{figure}

\newpage

\section{Einfluss der Positionierung der Kurzschl\"usse}
F\"ur diese Analyse werden 4 K\"urzschl\"usse einmal um 30 Grad versetzt um den Ringkern platziert, und einmal um 90 Grad versetzt.
\begin{figure}[h]
	\centering
	\begin{tikzpicture}
		\begin{axis}[ymode=log, width=0.85\textwidth, height = 0.5\textwidth, xmin = 0.1, xmax = 100, xlabel=Frequenz in MHz, ylabel=Realteil der Impedanz der Kavit\"at in Ohm, xticklabel style={/pgf/number format/fixed,/pgf/number format/precision=5}, every axis plot/.append style={thick},every axis legend/.append style={at={(0.02,0.97)},anchor=north west}, grid=both, cycle list name=color list]
			\addplot table[x index=0,y index=1,mark=none] {../Inputfiles/plotData/Box.txt};
			\addplot table[x index=0,y index=1,mark=none] {../Inputfiles/plotData/Rk.txt};
			\addplot table[x index=0,y index=1,mark=none] {../Inputfiles/plotData/Rk4Ks30.txt};
			\addplot table[x index=0,y index=1,mark=none] {../Inputfiles/plotData/Rk4Ks90.txt};
			\legend{leere Box, Box mit Ringkern, 4 Kurzschl\"usse (30 Grad versetzt), 4 Kurzschl\"usse (90 Grad versetzt)}
		\end{axis}
	\end{tikzpicture}
	\caption{Verhaltend der Box ohne Ringkern im Vergleich zur Box mit Ringkern, sowie mit mehreren Kurzschl\"ussen}
	\label{fig:PosKs}
\end{figure}

\newpage

\section{Einfluss der Form der Kurzschl\"usse}
F\"ur diese Analyse wird die Form der Kurzschl\"usse analysiert. Dazu wird wieder der einzelne Torus herangezogen und verglichen mit Verschieden breiten und weiten Kupferschienen. \\
\begin{figure}[h]
	\centering
	\begin{tikzpicture}
		\begin{axis}[ymode=log, width=0.85\textwidth, height = 0.5\textwidth, xmin = 0.1, xmax = 100, xlabel=Frequenz in MHz, ylabel=Realteil der Impedanz der Kavit\"at in Ohm, xticklabel style={/pgf/number format/fixed,/pgf/number format/precision=5}, every axis plot/.append style={thick},every axis legend/.append style={at={(0.02,0.97)},anchor=north west}, grid=both, cycle list name=color list]
			\addplot table[x index=0,y index=1,mark=none] {../Inputfiles/plotData/Box.txt};
			\addplot table[x index=0,y index=1,mark=none] {../Inputfiles/plotData/Rk.txt};
			\addplot table[x index=0,y index=1,mark=none] {../Inputfiles/plotData/Rk1Ks0.txt};
			\addplot table[x index=0,y index=1,mark=none] {../Inputfiles/plotData/Kupferschiene.txt};
			\addplot table[x index=0,y index=1,mark=none] {../Inputfiles/plotData/KupferschieneSchmal.txt};
			\legend{leere Box, Box mit Ringkern, 1 Kurzschluss(Torus), 1 Kupferschiene, 1 Kupferschiene (schmal)}
		\end{axis}
	\end{tikzpicture}
	\caption{Verhaltend der Box ohne Ringkern im Vergleich zur Box mit Ringkern, sowie mit mehreren Kurzschl\"ussen}
	\label{fig:FormKs}
\end{figure}

\newpage

Des Weiteren wird der Vergleich mit mehreren Kurzschl\"ussen gezogen. Hierbei werden 4 Toruskurzschl\"usse 4 Kupferschienenkurzschl\"ussen gegen\"uber gestellt.
\begin{figure}[h]
	\centering
	\begin{tikzpicture}
		\begin{axis}[ymode=log, width=0.85\textwidth, height = 0.5\textwidth, xmin = 0.1, xmax = 100, xlabel=Frequenz in MHz, ylabel=Realteil der Impedanz der Kavit\"at in Ohm, xticklabel style={/pgf/number format/fixed,/pgf/number format/precision=5}, every axis plot/.append style={thick},every axis legend/.append style={at={(0.02,0.97)},anchor=north west}, grid=both, cycle list name=color list]
			\addplot table[x index=0,y index=1,mark=none] {../Inputfiles/plotData/Box.txt};
			\addplot table[x index=0,y index=1,mark=none] {../Inputfiles/plotData/Rk.txt};
			\addplot table[x index=0,y index=1,mark=none] {../Inputfiles/plotData/Rk4Ks90.txt};
			\addplot table[x index=0,y index=1,mark=none] {../Inputfiles/plotData/Kupferschiene4x.txt};
			\legend{leere Box, Box mit Ringkern, 4 Kurzschl\"usse(Torus), 4 Kupferschienen}
		\end{axis}
	\end{tikzpicture}
	\caption{Verhaltend der Box ohne Ringkern im Vergleich zur Box mit Ringkern, sowie mit mehreren Kurzschl\"ussen}
	\label{fig:Form4Ks}
\end{figure}

\newpage

\section{Einfluss des Abstands der Kurzschl\"usse vom Ringkern}
\begin{figure}[h]
	\centering
	\begin{tikzpicture}
		\begin{axis}[ymode=log, width=0.85\textwidth, height = 0.5\textwidth, xmin = 0.1, xmax = 100, xlabel=Frequenz in MHz, ylabel=Realteil der Impedanz der Kavit\"at in Ohm, xticklabel style={/pgf/number format/fixed,/pgf/number format/precision=5}, every axis plot/.append style={thick},every axis legend/.append style={at={(0.32,0.4)},anchor=north west}, grid=both, cycle list name=color list]
			\addplot table[x index=0,y index=1,mark=none] {../Inputfiles/plotData/Box.txt};
			\addplot table[x index=0,y index=1,mark=none] {../Inputfiles/plotData/Rk.txt};
			\addplot table[x index=0,y index=1,mark=none] {../Inputfiles/plotData/Kupferschiene.txt};
			\addplot table[x index=0,y index=1,mark=none] {../Inputfiles/plotData/KupferschieneAbstandsvariation.txt};
			\addplot table[x index=0,y index=1,mark=none] {../Inputfiles/plotData/KupferschieneEngRK.txt};
			\legend{leere Box, Box mit Ringkern, 1 Kupferschiene, 1 Kupferschiene (weiter Abstand von Ringkern, 1 Kupferschiene (eng am Ringkern)}
		\end{axis}
	\end{tikzpicture}
	\caption{Verhaltend der Box ohne Ringkern im Vergleich zur Box mit Ringkern, sowie mit mehreren Kurzschl\"ussen}
	\label{fig:FormKs}
\end{figure}

\newpage

\section{Einfluss im Falle einer passiven Schiene}
Bei einer Schiene:
\begin{figure}[h]
	\centering
	\begin{tikzpicture}
		\begin{axis}[ymode=log, width=0.85\textwidth, height = 0.5\textwidth, xmin = 0.1, xmax = 100, xlabel=Frequenz in MHz, ylabel=Realteil der Impedanz der Kavit\"at in Ohm, xticklabel style={/pgf/number format/fixed,/pgf/number format/precision=5}, every axis plot/.append style={thick},every axis legend/.append style={at={(0.02,0.97)},anchor=north west}, grid=both, cycle list name=color list]
			\addplot table[x index=0,y index=1,mark=none] {../Inputfiles/plotData/Rk.txt};
			\addplot table[x index=0,y index=1,mark=none] {../Inputfiles/plotData/KupferschieneOffen.txt};
			\legend{Box mit Ringkern, Box mit einer Offenen Kupferschiene}
		\end{axis}
	\end{tikzpicture}
	\caption{Verhaltend der Box mit Ringkern im Vergleich zur Box mit einer offenen Kupferschiene}
	\label{fig:passiv}
\end{figure}

\newpage

Bei mehreren Schienen:
\begin{figure}[h]
	\centering
	\begin{tikzpicture}
		\begin{axis}[ymode=log, width=0.85\textwidth, height = 0.5\textwidth, xmin = 0.1, xmax = 100, xlabel=Frequenz in MHz, ylabel=Realteil der Impedanz der Kavit\"at in Ohm, xticklabel style={/pgf/number format/fixed,/pgf/number format/precision=5}, every axis plot/.append style={thick},every axis legend/.append style={at={(0.02,0.97)},anchor=north west}, grid=both, cycle list name=color list]
			\addplot table[x index=0,y index=1,mark=none] {../Inputfiles/plotData/Rk.txt};
			\addplot table[x index=0,y index=1,mark=none] {../Inputfiles/plotData/KupferschieneOffen4x.txt};
			\legend{Box mit Ringkern, Box mit vier offenen Kupferschienen}
		\end{axis}
	\end{tikzpicture}
	\caption{Verhaltend der Box mit Ringkern im Vergleich zur Box mit einer offenen Kupferschiene}
	\label{fig:passiv}
\end{figure}