\section{Feldbilder}
Um eine weitergehende Analyse der Auswirkung von Kurzschl\"ussen zu f\"uhren, k\"onnen aus CST Feldbilder ausgelesen werden. Diese zeigen, in wie weit das magnetische Feld durch die Kurzschl\"usse aus dem inneren des MA-Ringkerns verdr\"angt wird. Dazu werden einige Kurzschlussanordnungen gegen\"uber gestellt. Zun\"achst wird der Einfluss eines einzigen Kurzschlusses mit der Breite $\SI{30}{\milli\meter}$, der H\"ohe $\SI{160}{\milli\meter}$ sowie einer Blechdicke von $\SI{1}{\milli\meter}$ gegen\"uber dem reinen Ringkern ohne Kurzschl\"usse nach Abbildung~\ref{fig:0zu1ks} betrachtet.
\todo[inline,color=red!30]{Grafik oKS zu 1KS einf\"ugen}
\par
Bereits ein Kurzschluss verd\"angt schon einen gro\ss{}en Teil des magnetischen Feldes. Dieser Effekt ist in dem Bereich des Ringkerns, welcher von der Kurzschlusschiene \"uberdeckt ist, oder in der n\"ahe liegt deutlich st\"arker als in anderen Ringkernregionen. Dieser Effekt ist auch bei einer h\"oheren Anzahl an Kurzschl\"ussen sichtbar. Abbildung~\ref{fig:1zu8ks} zeigt den Vergleich von einem Kurzschluss zur maximalen Anzahl von acht Kurzschl\"ussen.
\todo[inline,color=red!30]{Grafik 1KS zu 8KS einf\"ugen}
\par
Die Beaobachtung liefert auch eine Erkl\"arung, warum zwei schmale Kurzschl\"usse eine st\"arkere Verringerung der Ringkernimpedanz nach sich ziehen, als ein breiter Kurzschluss. Da das Feld besonders im Umkreis des Kurzschlusses geringer ist, f\"uhrt eine weitere Verteilung der Kurzschl\"usse zu geringeren Feldst\"arken im gesamten Ring. Dieser Effekt ist in Abbildung~\ref{fig:150zu220ks} zu sehen. 
\todo[inline,color=red!30]{Grafik 1KS 50mm breite zu 2KS 20mm Breite einf\"ugen}
\par
Wird vorausgesetzt, dass die Ringkernimedanz direkt mit dem mittleren Feld zusammenh\"angt, so dient dies als eine plausible Erkl\"arung f\"ur bisher beobachtete Effekte. Die Komplette Ansicht der Feldverteilung f\"ur alle Kurzschlussanordnungen ist in Anhang~\ref{sec:allfieldplots} gegeben.