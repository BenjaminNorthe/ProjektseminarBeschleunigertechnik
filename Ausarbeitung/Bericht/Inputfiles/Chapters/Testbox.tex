\section{Testbox}
Der f\"ur Messungen verwendete Testaufbau konnte aus dem Projekt von Michael Frey und Jens Harzheim [\textbf{QUELLE EINF\"UGEN}] \"ubernommen werden. Ziel des Testaufbaus war es, eine reproduzierbare Vermessung des Einflusses der Magnetic-Alloy-Ringkerne auf die Impedanz einer Einkopplung zu erreichen. Dadurch soll eine Absch\"atzung des Einfluss auf die Strahlimpedanz in der Kavit\"at erm\"oglicht werden. Im Rahmen der Bachelorarbeit von Denys Bast wurde f\"ur diese Testbox auch ein Simulationsmodell erstellt. Dieses wird in Abschnitt~\ref{ch:sim} behandelt. 
\par
Ausgehend von den bestehenden Aufbauten und Modellen wird im folgenden Analysiert, inwiefern ein oder mehrere Sekund\"are Kurzschl\"usse die Feldimpedanz\"anderung des Ringkerns annulieren k\"onnen. 

\subsection{Anfangsmessung}
Um eine Grobe Tendenz und ein Gef\"uhl f\"ur den Messaufbau zu erreichen wurden zun\"acht einige Messungen an der unmodifizierten Testbox ausgef\"uhrt.

\subsection{Modifikation}