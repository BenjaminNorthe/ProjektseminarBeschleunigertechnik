\chapter*{Zusammenfassung}
Die Shuntimpedanz von breitbandigen Kavitätensystemen kann je nach Beschleunigungszyklus einen hohen Einfluss auf den Teilchenstrahl haben und diesem Energie entziehen. Im Rahmen des Projektseminars wurde untersucht, inwieweit sich die effektive Impedanz von Kavitäten reduzieren lässt, indem eine oder mehrere zusätzliche Kurzschlusswindungen um die in den Kavitäten verbauten Ringkerne geschlossen werden. Dabei wurden verschiedene geometrische Parameter mittels Messungen und CST-Simulationen getestet und verglichen, mit dem Ziel, eine möglichst effektive Reduktion der Impedanz zu erreichen.