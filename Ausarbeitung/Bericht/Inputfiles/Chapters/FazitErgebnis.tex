\section{Fazit}
was wurde gemacht:
\begin{itemize}
	\item Testbox vermessen
	\item Testbox modifiziert um reproduzierbar zu messen
	\item Messung durchgef\"uhrt, reproduzierbar
	\item Simulation angepasst, dass diese in geringer Abweichung f\"ur weitere Variationen genutzt werden kann
	\item Ergebnisse quantifiziert, evaluiert
\end{itemize}
was wurde beobachtet:
\begin{itemize}
    \item ein KS bringt schon eine Annullierung von ca. $80~\%$
    \item mit 7 KS fast kein Einfluss durch RK mehr
    \item Breite, Länge und Dicke haben vergleichbar geringen Einfluss
\end{itemize}
welche Schlüsse werden daraus gezogen:
\begin{itemize}
    \item je nach einbaumöglichkeit besser schmalere, kurze, nah um den RK und passende Anzahl wählen
\end{itemize}

\section{Ausblick}
\begin{itemize}
    \item verbesserung der Simulationsmodell mit besserer anpassung der materialparameter
    \item anpassung der parameter an die Umgebung der kavität
    
\end{itemize}

aus anderen Kapiteln genommen:
Anzahl:
Je nach Anforderung ist also zu \"uberlegen, ob ein Kurzschluss bereits eine ausreichende Reduktion der Impedanz erzeugt. Das ist insbesondere beim Einbau in die Kavit\"at von Relevanz, da eine Montage mehrerer Kurzschl\"usse einen nicht unerheblichen Aufwand mit sich zieht. 

Breite:
Sollte aus Platzgr\"unden eine Montage breiterer Kurzschl\"usse in der Kavit\"at zu Problemen f\"uhren, so ist eine h\"ohere Anzahl von schmaleren Kurzschl\"ussen vorzuziehen. 

Länge:
Das bedeutet, dass eine geringere L\"ange der Kurzschl\"usse bessere Ergebnisse erzeugt, und diese folglich nach M\"oglichkeit klein sein sollte. Allerdings f\"allt die Auswirkung sehr gering aus, weshalb dieser Parameter eher niedrige Priorit\"at erhalten sollte, falls der Einbau durch eine zu kleine L\"ange der Kurzschl\"usse eingeschr\"ankt wird.