\section{Fazit}
Diese Arbeit untersucht den Einfluss von Kurzschlüssen auf die Impedanz von Ringkernen, die in Kavitäten als Last eingesetzt werden. Damit dem Strahl durch die Ringkerne nur ein geringer Teil Energie entzogen wird, wenn die Kavität nicht beschleunigt, sollen die Kurzschlüsse die Ringkernimpedanz verringern. Es wurde eine Parameteranalyse durchgeführt, die ermittelt welche Parameter einen Einfluss auf die Impedanz haben und wie groß dieser ausfällt.\\
Zu diesem Zweck wurden Messungen an einer Testbox durchgeführt, die eine reproduzierbare Vermessung der Ringkerne gewährleistet. Außerdem wurden Simulationen der verwendeten Anordnung vorgenommen. Aus den Ergebnissen der Messungen und Simulationen wurden Anpassungen für die Testbox abgeleitet und Kurzschlüsse für reproduzierbare Messungen erstellt.\\
Für die Auswertung der Mess- und Simulationsergebnisse wurde die Simulation schließlich noch besser an die Realität angepasst und die Ergebnisse gegenübergestellt und evaluiert.
\par
Die Auswertung und Beurteilung der Ergebnisse der Simulation und Messung ergaben, dass bereits ein Kurzschluss die Ringkernimpedanz um ungefähr $80\,\%$ verringert. Wird der Ringkern mit sieben Schienen kurzgeschlossen, so werden mehr als $98\,\%$ der Ringkernimpedanz annulliert.\\
Es wurde außerdem festgestellt, dass die Länge, die Breite und die Dicke der verwendeten Kurzschlüsse ebenfalls einen Einfluss auf die Ringkernimpedanz ausüben. Die erreichte Verringerung fällt im Vergleich zu einer Variation der Anzahl der Kurzschlüsse wesentlich geringer aus.
\par
Sollen im realen Betrieb die Ringkerne einer Kavität also kurzgeschlossen werden, so ist die Wahl der Kurzschlüsse vornehmlich vom vorhanden Platzangebot in der Kavität bestimmt.\\
Es ist also abzuwägen, ob mit nur einem Kurzschluss bereits genügend Impedanz reduziert werden kann oder ob noch weitere nötig sind. Die Breite und die Länge haben als Paramter nur einen kleinen Einfluss. Für die Anwendung empfiehlt es sich daher auf schmale, eng anliegende Kurzschlüsse zurückzugreifen. 
\par

\section{Ausblick}
Das Simulationsmodell bietet weitere Verbesserungsm\"oglichkeiten. Diese beziehen sich vorwiegend auf die verwendeten Materialparameter.
In zuk\"unftigen Arbeiten k\"onnte daher die Wertetabelle des dissipativen Holzmaterials(siehe Kapitel \ref{sec:gegenueberst} und \ref{sec:realanpassung}) mittels einer Kreuzvalidierung weiter angepasst werden. Zur Rechtfertigung dieses Schrittes k\"onnten die Materialparameter der Testbox auch experimentell ermittelt werden.
Um eine noch bessere \"Ubereinstimmung zwischen dem Simulationsmodell und den Messungen erreichen zu k\"onnen, empfiehlt es sich f\"ur zuk\"unftige Arbeiten d
An dieser Stelle werden nun m\"ogliche Verbesserungen des Simulationsmodells 
\begin{itemize}
    \item verbesserung der Simulationsmodell mit besserer anpassung der materialparameter
    \item anpassung der parameter an die Umgebung der kavität
    \item Messungen an der kavität selbst
    \item ausarbeitung für schließen und öffnen
\end{itemize}