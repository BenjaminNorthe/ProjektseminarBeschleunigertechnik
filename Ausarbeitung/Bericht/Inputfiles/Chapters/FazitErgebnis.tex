\section{Fazit}
Diese Arbeit untersucht den Einfluss von Kurzschlüssen auf die Impedanz von Ringkernen, die in Kavitäten als Last eingesetzt werden. Damit dem Strahl durch die Ringkerne nur ein geringer Teil Energie entzogen wird, wenn die Kavität nicht beschleunigt, sollen die Kurzschlüsse die Ringkernimpedanz verringern. Es wurde eine Parameteranalyse durchgeführt, die ermittelt welche Parameter einen Einfluss auf die Impedanz haben und wie groß dieser ausfällt.\\
Zu diesem Zweck wurden Messungen an einer Testbox durchgeführt, die eine reproduzierbare Vermessung der Ringkerne gewährleistet. Außerdem wurden Simulationen der verwendeten Anordnung vorgenommen. Aus den Ergebnissen der Messungen und Simulationen wurden Anpassungen für die Testbox abgeleitet und Kurzschlüsse für reproduzierbare Messungen erstellt.\\
Für die Auswertung der Mess- und Simulationsergebnisse wurde die Simulation schließlich noch besser an die Realität angepasst und die Ergebnisse gegenübergestellt und evaluiert.
\par
Die Auswertung und Beurteilung der Ergebnisse der Simulation und Messung ergaben, dass bereits ein Kurzschluss die Ringkernimpedanz um ungefähr $80\,\%$ verringert. Wird der Ringkern mit sieben Schienen kurzgeschlossen, ist die Ringkernimpedanz schon fast komplett annulliert.\\
Es wurde außerdem festgestellt, dass die Länge, die Breite und die Dicke der verwendeten Kurzschlüsse ebenfalls einen Einfluss auf die Ringkernimpedanz ausüben. Die erreichte Verringerung fällt im Vergleich zu einer Variation der Anzahl der Kurzschlüsse wesentlich geringer aus.
\par
Sollen im realen Betrieb die Ringkerne einer Kavität also kurzgeschlossen werden, so ist die Wahl der Kurzschlüsse vornehmlich vom vorhanden Platzangebot in der Kavität bestimmt.\\
Es ist also abzuwägen, ob mit nur einem Kurzschluss bereits genügend Impedanz reduziert werden kann oder ob noch weitere nötig sind. Da die Breite und die Länge einen nicht so großen Einfluss haben, kann man hierbei auf schmale Kurzschlüsse zurückgreifen, die enger um die Ringkerne angebracht werden, insbesondere da sich gezeigt hat ein kurzer Kurzschluss bessere Ergebnisse lieferte.
\par 
was wurde gemacht:
\begin{itemize}
    \item Testbox vermessen
    \item Testbox modifiziert um reproduzierbar zu messen
    \item Messung durchgef\"uhrt, reproduzierbar
    \item Simulation angepasst, dass diese in geringer Abweichung f\"ur weitere Variationen genutzt werden kann
    \item Ergebnisse quantifiziert, evaluiert
\end{itemize}
was kam heraus:
\begin{itemize}
    \item ein KS bringt schon eine Annullierung von ca. $80~\%$
    \item mit 7 KS fast kein Einfluss durch RK mehr
    \item Breite, Länge und Dicke haben vergleichbar geringen Einfluss
\end{itemize}
welche Schlüsse werden daraus gezogen:
\begin{itemize}
    \item je nach einbaumöglichkeit besser schmalere, kurze, nah um den RK und passende Anzahl wählen
\end{itemize}
\par 
aus anderen Kapiteln genommen:
Anzahl:
Je nach Anforderung ist also zu \"uberlegen, ob ein Kurzschluss bereits eine ausreichende Reduktion der Impedanz erzeugt. Das ist insbesondere beim Einbau in die Kavit\"at von Relevanz, da eine Montage mehrerer Kurzschl\"usse einen nicht unerheblichen Aufwand mit sich zieht. 

Breite:
Sollte aus Platzgr\"unden eine Montage breiterer Kurzschl\"usse in der Kavit\"at zu Problemen f\"uhren, so ist eine h\"ohere Anzahl von schmaleren Kurzschl\"ussen vorzuziehen. 

Länge:
Das bedeutet, dass eine geringere L\"ange der Kurzschl\"usse bessere Ergebnisse erzeugt, und diese folglich nach M\"oglichkeit klein sein sollte. Allerdings f\"allt die Auswirkung sehr gering aus, weshalb dieser Parameter eher niedrige Priorit\"at erhalten sollte, falls der Einbau durch eine zu kleine L\"ange der Kurzschl\"usse eingeschr\"ankt wird.

\section{Ausblick}
\begin{itemize}
    \item verbesserung der Simulationsmodell mit besserer anpassung der materialparameter
    \item anpassung der parameter an die Umgebung der kavität
    
\end{itemize}