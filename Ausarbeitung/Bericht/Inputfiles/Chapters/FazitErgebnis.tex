\section{Fazit}
was wurde gemacht:
\begin{itemize}
	\item Testbox vermessen
	\item Testbox modifiziert um reproduzierbar zu messen
	\item Messung durchgef\"uhrt, reproduzierbar
	\item Simulation angepasst, dass diese in geringer Abweichung f\"ur weitere Variationen genutzt werden kann
	\item Ergebnisse quantifiziert, evaluiert
\end{itemize}
was wurde beobachtet:
\begin{itemize}
    \item ein KS bringt schon eine Annullierung von ca. $80~\%$
    \item mit 7 KS fast kein Einfluss durch RK mehr
    \item Breite, Länge und Dicke haben vergleichbar geringen Einfluss
\end{itemize}
welche Schlüsse werden daraus gezogen:
\begin{itemize}
    \item je nach einbaumöglichkeit besser schmalere, kurze, nah um den RK und passende Anzahl wählen
\end{itemize}

\section{Ausblick}
\begin{itemize}
    \item verbesserung der Simulationsmodell mit besserer anpassung der materialparameter
    \item anpassung der parameter an die Umgebung der kavität
    
\end{itemize}