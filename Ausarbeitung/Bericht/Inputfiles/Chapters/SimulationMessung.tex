\section{Gegen\"uberstellung der Simulations und Messergebnisse}
Ausgehend von den beschriebenen Modellierungsschritten kann nun eine sichere Evaluierung der Kurzschlussanordnungen angesetzt werden. Dabei k\"onnen die Messungen sowie das Simulationsmodell zur Kreuzvalidierung verwendet werden, sodass Mess- und Simulationsfehler weitestgehend auszuschlie\ss{}en sind. Dazu wurde das Simulationsmodell, nach den in Absatz~\ref{ch:sim} versehenen Anpassungen zun\"ach einmal zur Referenz mit den Messungen verglichen. Dazu wird zun\"achst die Testbox ohne Ringkern, jedoch mit fertigem Halterungsaufbau gegen\"ubergestellt. Abbildung~\ref{fig:boxpolycross} zeigt diese Gegen\"uberstellung.
\begin{figure}[htb]
	\centering
	\includegraphics[width=0.95\textwidth]{boxpolycrossplot}
	\caption{Gegen\"uberstellung der Simulation der Box mit Halterung aus Kreuz und Polygon zur entsprechenden Messung.}
	\label{fig:boxpolycross}
\end{figure}
