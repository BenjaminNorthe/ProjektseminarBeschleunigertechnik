\section{Testbox und Material}
Wie in Abschnitt~\ref{sec:testbox} angef\"uhrt, kann das Modell f\"ur die Simulationen in Grundz\"ugen aus der Bachelorarbeit von Denys Bast~\cite{bast2017ba} \"ubernommen werden. Die Modellierung der Testbox mitsamt Ringkern wurde dabei in mehreren Schritten durchgef\"uhrt. Die W\"ande der Testbox wurden mit idealem Kupfer modelliert. Ebenso wurde die Einkopplung und der Innenleiter durch die Box mit idealem Kupfer dargestellt und die Dimensionen der Verbindung entsprechend angepasst. Die Holzkonstruktion zur Halterung der Rinkerne ist im Urspr\"unglichen Modell nicht modelliert, der Ringkern h\"angt dort also in der Luft.
\par
Der Ringkern selbst, beziehungsweise das Ringkernmaterial wurde aus der Messung herausgezogen. Dazu wurde die Messkurve der Ringkernbox als RLC-Modellierung gefittet. Aus den berechneten Werte f\"ur L und C k\"onnen anschlie\ss{}end die Parameter $\mu'$ und $\mu''$ mit Hilfe der Formeln $1$ bis $3$ berechnet werden.
\par
\begin{figure}[htb]
\centering
\begin{tikzpicture}
	\begin{circuitikz}
	\draw
	(2,0) to [resistor =$R$] (2,3)
	(5,0) to [C, l=$C$] (5,3)
	(8,0) to [L, l=$L$] (8,3)
	(0,3) to [short, *-] (8,3)
	(0,0) to [short, *-] (8,0);
% 		(0,0) to [short, *- , i_=$I_5$] (1.5,-2);
	\end{circuitikz}
\end{tikzpicture}
\caption{RLC-Ersatzschaltbild f\"ur die Testbox Modellierung.}
\end{figure}

\par

\begin{figure}[htb]
\centering
\begin{tikzpicture}
	\begin{circuitikz}
	\draw
	(2,0) to [resistor =$R$] (2,3)
	(5,0) to [C, l=$C$] (5,3)
	(5,3) to [L, l=$L$] (9,3)
	(9,3) to [short, *-] (10,3)
	(9,0) to [short, *-] (10,0)
	(10,3) to [resistor =$R_{last}$] (10,0)
	(0,3) to [short, *-] (5,3)
	(0,0) to [short, *-] (5,0)
	(8,3) to [short, -*] (9,3)
	(5,0) to [short, -*] (9,0);
% 		(0,0) to [short, *- , i_=$I_5$] (1.5,-2);
	\end{circuitikz}
\end{tikzpicture}
\caption{RLC-Ersatzschaltbild f\"ur die Testbox Modellierung mit Eingebrachtem Ringkern als Last.}
\end{figure}

