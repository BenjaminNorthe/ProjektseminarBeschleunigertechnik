\section{Aufgabenstellung}
%     \begin{enumerate}
%         \item Untersuchung verschiedener Parameter von Kurzschlüssen um Ringkerne und deren Einfluss auf die Impedanz
%     \end{enumerate}
Die Impedanz der Ringkerne wirkt sich auf den Teilchenstrahl im Beschleuniger aus. Diese Auswirkung soll möglichst reduziert werden, wenn die Kavität den Teilchenstrahl nicht manipulieren.
\par
Die Aufgabe des Projektseminars besteht deshalb darin, zu analysieren, in wie weit das Kurzschlie\ss{}en der MA-Ringkerne innerhalb der Kavit\"at die Impedanz dieser verringern kann. Dazu sind mehrere Parameter der Kurzschl\"usse, sowie deren Anzahl zu untersuchen.
\par
\textcolor{blue}{
Damit die Ergebnisse auch mit anderen Parametern untersucht werden k\"onnen, wird der gesamte Aufbau samt der Halterung f\"ur den Ringkern, sowie der Kurzschl\"usse, ausgehend von dem bereits bestehenden Modell aus der Bachelorarbeit von Denys Bast~\citep{bast2017ba}, in CST modelliert.}
\par
\textcolor{blue}{
Abschlie\ss{}end werden sowohl die Simulationsergebnisse, als auch die Messergebnisse unter \"Anderung der Paramter gegen\"ubergestellt. Dadurch kann eine pr\"aferierte Anordnung zum Kurzschlie\ss{}en der MA-Ringkerne bestimmt werden. Diese Schritte werden in den folgenden Kapiteln erl\"autert.}
\par 
\textcolor{red}{Alternativ:}

