\section{Aufgabenstellung}
%     \begin{enumerate}
%         \item Untersuchung verschiedener Parameter von Kurzschlüssen um Ringkerne und deren Einfluss auf die Impedanz
%     \end{enumerate}
Soll die Kavit\"at station\"ar betrieben, also keine Beschleunigung ausgef\"uhrt werden, so ist die Impedanz der Ringkerne ein Problem, da diese eine Auswirkung auf den Teilchenstrahl im Beschleuniger auswirkt.
\par
Die Aufgabe des Projektseminars besteht deshalb darin, zu analysieren, in wie weit das Kurzschlie\ss{}en der MA-Ringkerne innerhalb der Kavit\"at die Impedanz dieser verringern kann. Dazu sind mehrere Parameter der Kurzschl\"usse, sowie deren Anzahl zu untersuchen.
\par
Damit die Ergebnisse auch mit anderen Parametern untersucht werden k\"onnen, wird der gesamte Aufbau samt der Halterung f\"ur den Ringkern, sowie der Kurzschl\"usse, ausgehend von dem bereits bestehenden Modell aus der Bachelorarbeit von Denys Bast~\cite{bast2017ba}, in CST modelliert. 
\par
Abschlie\ss{}end werden sowohl die Simulationsergebnisse, als auch die Messergebnisse unter \"Anderung der Paramter gegen\"ubergestellt um eine Bewertung einer pr\"aferierten Anordnung zum Kurzschlie\ss{}en der MA-Ringkerne zu bestimmen. Diese Schritte werden in den folgenden Kapiteln erleutert. 