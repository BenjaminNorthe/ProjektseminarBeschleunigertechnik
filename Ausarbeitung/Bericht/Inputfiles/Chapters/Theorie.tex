\section{Theorie}

Bei der Gesellschaft für Schwerionenforschung(GSI) werden in der Gruppe Ring-RF Systems unter anderem  Barrier Bucket Systeme entwickelt. Diese können zur Modifikation der longitudinalen Strahldynamik, wie etwa zum Zusammenf\"uhren oder Verdichten eines oder mehrerer Strahl-Bunches verwendet werden. 
Das Grundprizip der Barrier Bucket Systeme besteht, darin, Teilchen zwischen zwei Potentialbarrieren zu begrenzen. Diese Potentialbarrieren k\"onnen verschiedene Formen aufweisen, h\"aufig werden einzelne Sinus-Pulse verwendet~\citep{harzheim2016modeling}~\citep{lee1997particle}. Durch \"Anderung der Barrieren, wie etwa der Variation des Abstandes zwischen einzelnen Barrieren, k\"onnen Somit die Teilchen verdichtet oder verteilt werden.
Die Barrier Bucket Systeme Systeme verwenden unter anderem magnetische Ringkerne als Last. Vorteil ist, dass mit diesem Aufbau ein Tuning der Resonanzfrequenz m\"oglich ist (siehe Kapitel 4 in \citep{Klingbeil2015} bzw. Kapitel 2 in \citep{bast2017ba}).
\par
Die MA(magnetic alloy)-Ringkerne weisen dabei eine charakteristische Impedanz auf, welche nicht einer idealen Spule entspricht, sondern vielmehr ein dissipatives Verhalten zeigt. Eine genauere Herleitung kann in \citep{Klingbeil2015} gefunden werden und liefert folgenden Zusammenhang. 


\begin{align}
Z_{rk} = R_{rk} + j\omega L_{rk}\label{eq_01}\\
L_{rk} = \frac{Nd\mu^\prime}{2\pi}\cdot ln(\frac{r_o}{r_i})\label{eq_02}\\
R_{rk} = \omega L_{rk} \frac{\mu^{\prime\prime}}{\mu^{\prime}}\label{eq_03}
\end{align}



