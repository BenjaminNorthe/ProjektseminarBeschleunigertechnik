\section{Theorie}

Bei der Gesellschaft für Schwerionenforschung(GSI) werden in der Gruppe Ring-HF Systems Beschleunigerkomponenten entwickelt. Unter anderem werden Barrier Bucket Systeme entwickelt, welche zum zusammenf\"uhren oder verdichten eines oder mehrerer Strahl-Bunches verwendet werden k\"onnen. Diese Systeme verwenden unter anderem magnetische Ringkerne verwenden. Vorteil ist, dass mit diesem Aufbau ein Tuning der Resonanzfrequenz m\"oglich ist (siehe Kapitel 4 in \cite{Klingbeil2015} bzw. Kapitel 2 in \cite{bast2017ba}).\\

Die MA(magnetic alloy)-Ringkerne weisen dabei eine charakteristische Impedanz auf, welche nicht einer idealen Spule entspricht, sondern vielmehr ein dissipatives Verhalten zeigt. Eine genauere Herleitung kann in \cite{Klingbeil2015} gefunden werden und liefert folgenden Zusammenhang. 

\begin{align}
Z_{rk} = R_{rk} + j\omega \L_{rk}\\
L_{rk} = \frac{N\cdot d\cdot\mu^\prime}{2\cdot\pi}\cdot ln(\frac{r_o}{r_i})\\
R_{rk} = \omega\cdot L_{rk}\cdot\frac{\mu^{\prime\prime}}{\mu^{\prime}}
\end{align}

 