\documentclass[accentcolor=tud9b, colorbacktitle]{tudbeamer}

\usepackage[utf8]{inputenc}
\usepackage[T1]{fontenc}
\usepackage[ngerman]{babel}
\usepackage{epstopdf}
\usepackage{graphicx}
\usepackage{xcolor}
\usepackage{tikz}
\usepackage{lipsum,multicol}


\usepackage{tikz}						% graphics creation environment frontend
\usepackage{tikz-cd}
\usepackage{import}
\usepackage{multicol}
% 
\usepackage{pgfplots}					% graph plotting for pgf/tikz
\pgfplotsset{compat=1.12, grid style={gray,dotted}}

\usetikzlibrary{external} %% comment out to stop externalization of tikz pictures
\tikzexternalize[optimize=false, prefix=tikz-external/] % path to store the externalized stuff in
\tikzset{external/system call={lualatex \tikzexternalcheckshellescape -halt-on-error-interaction=batchmode -jobname "\image" "\texsource"}, force remake = false}
% \tikzset{external/force remake = false}
\tikzexternalize

\graphicspath{{graph/}}

\author{Julian Buschbaum, Rainer Stellnberger, Benjamin Northe}
\institute{Institut TEMF}
\logo[1.5]{\includegraphics{temf}}
\date{\today}
\title{}

\begin{document}

\begin{titleframe}


\end{titleframe}





\begin{frame}\frametitle{Inhalt}

\end{frame}


\begin{frame}\frametitle{Simulationen}
Mithilfe von CST wurden mehrere Faktoren f\"ur die Kurzschl\"usse durchsimuliert:
\begin{itemize}
	\item Feldimpedanz f\"ur verschiedene Anzahlen an Kurzschl\"ussen
	\item Feldimpedanz f\"ur verschiedene Positionen der Kurzschl\"usse
	\item Feldimpedanz f\"ur verschiedene Formen der Kurzschl\"usse
	\item Feldimpedanz f\"ur verschiedene Abst\"ande der Kurzschlusswicklung zum Ringkern
	\item Feldimpedanz mit einer unterbrochenen Schiene (wenn sich diese im Leerlauf befindet)
\end{itemize}
\end{frame}



\begin{frame}\frametitle{Feldimpedanz f\"ur verschiedene Anzahlen an Kurzschl\"ussen}
\vspace{-2em}
\begin{figure}[h]
	\centering
	\begin{tikzpicture}
		\begin{axis}[ymode = log, width=0.85\textwidth, height = 0.58\textwidth, xmin = 0.1, xmax = 100, xlabel=Frequenz in MHz, ylabel=Realteil der Impedanz der Kavit\"at in Ohm, xticklabel style={/pgf/number format/fixed,/pgf/number format/precision=5}, every axis plot/.append style={thick},every axis legend/.append style={at={(0.3,0.45)},anchor=north west}, grid=both, cycle list name=color list]
			\addplot table[x index=0,y index=1,mark=none] {graph/plotData/Box.txt};
			\addplot table[x index=0,y index=1,mark=none] {graph/plotData/Rk.txt};
			\addplot table[x index=0,y index=1,mark=none] {graph/plotData/Rk1Ks0.txt};
			\addplot table[x index=0,y index=1,mark=none] {graph/plotData/Rk4Ks90.txt};
			\addplot table[x index=0,y index=1,mark=none] {graph/plotData/Rk24Ks15.txt};
			\legend{leere Box, Box mit Ringkern, 1 Kurzschluss, 4 Kurzschl\"usse (90 Grad versetzt),  24 Kurzschl\"usse}
		\end{axis}
	\end{tikzpicture}
\end{figure}
\end{frame}



\begin{frame}\frametitle{Feldimpedanz f\"ur verschiedene Positionen der Kurzschl\"usse}
\vspace{-2em}
\begin{figure}[h]
	\centering
	\begin{tikzpicture}
		\begin{axis}[ymode=log, width=0.85\textwidth, height = 0.5\textwidth, xmin = 0.1, xmax = 100, xlabel=Frequenz in MHz, ylabel=Realteil der Impedanz der Kavit\"at in Ohm, xticklabel style={/pgf/number format/fixed,/pgf/number format/precision=5}, every axis plot/.append style={thick},every axis legend/.append style={at={(0.3,0.45)},anchor=north west}, grid=both, cycle list name=color list]
			\addplot table[x index=0,y index=1,mark=none] {graph/plotData/Box.txt};
			\addplot table[x index=0,y index=1,mark=none] {graph/plotData/Rk.txt};
			\addplot table[x index=0,y index=1,mark=none] {graph/plotData/Rk4Ks30.txt};
			\addplot table[x index=0,y index=1,mark=none] {graph/plotData/Rk4Ks90.txt};
			\legend{leere Box, Box mit Ringkern, 4 Kurzschl\"usse (30 Grad versetzt), 4 Kurzschl\"usse (90 Grad versetzt)}
		\end{axis}
	\end{tikzpicture}
\end{figure}
 
\end{frame}



\end{document}
